\documentclass[12pt,a4paper]{article}
%\usepackage{paracol}
%\usepackage{amsmath}
\usepackage{graphicx}
%\usepackage{fancyhdr}
%\pagestyle{fancy}
\usepackage[hmargin=2cm,vmargin=4.5cm]{geometry}
\begin{document}
\begin{center}
\section*{\textbf{\Huge MINI PROJECT REPORT }}
\section*{\textbf{\Large Objectives Of Project}}
\end{center}
\begin{itemize}
\item Develop programs for complex real world problems. 
\item Apply good programming practices in their code like Comments, indentation etc. 
\item Utilize Debugger and its tools like gdb/gnu for error handling.
\item Demonstrate configuration and usage of different software tools used in industry.
\end{itemize}
\pagebreak

\begin{center}
\section*{\textbf{\Large Function Description}}
\end{center}
\subsection*{Function 1 - hcf}
First function is hcf which takes 2 arguements of datatype int and prints the hcf of both the entered numbers.
\\
Hcf is calculated by subtracting the smaller number from the larger one and continuing this process until both the numbers became equal.
\\
The value at which both the numbers became equal is the required hcf. 
\\

\subsection*{Function 2 - palindrome}
Second function is palindrome which take 1 arguement of datatype int and prints if the entered number is a palindrome or not.
\\
A number is said to be palindrome if its original value is equal to its reverse value.
\\

\subsection*{Function 3 - perfect}
Third function is perfect which take 1 arguement of datatype int and prints if the entered number is a perfect number or not.
\\
A number is said to be perfect if sum of its divisors ,except the number itself, is equal to the number.
\\

\subsection*{Function 4 - prime}
Fourth function is prime which take 1 arguement of datatype int and prints if the entered number is a prime number or not.
\\
A number is said to be prime if it is divisible by only 1 and the number itself.
\\

\subsection*{Function 5 - armstrong}
Fifth function is armstrong which take 1 arguement of datatype int which is a three digit integer and prints if the entered number is a armstrong number or not.
\\
A three digit number is said to be armstrong number if sum of cubes of its digit is equal to the original number.
\\

\subsection*{Function 6 - factorial}
Sixth function is factorial which take 1 arguement of datatype int and prints factorial of entered number.
\\
Factorial of any number n is the product of first n natural numbers.
\\

\subsection*{Function 7 - sum}
Seventh function is sum which take 1 arguement of datatype int and prints the sum of first n natural numbers.
\\

\subsection*{Function 8 - coprime}
Eighth function is coprime which takes 2 arguement of datatype int and prints if enetered numbers are co-prime to each other or not.
\\
Two numbers are said to be co-prime if they have only one factor in common, that is, 1.
\\

\subsection*{Function 9 - power}
Ninth function is power which take 2 arguement of datatype int ,that are base and exponent, and prints the value of base to the power exponent.
\\

\subsection*{Function 10 - magic\_number}
Tenth function is magic\_number which take 1 arguement of datatype int and prints if the entered number is magic number or not.
\\
A number is said to be magic number if we sum its digits and again sum the digits of the number obtained until we reach a single digit.\\ If the single digit obtained is 1 the
number is said to be as magic number, otherwise the number is not a magic number. 
\\
\pagebreak
\begin{center}
\section*{\textbf{\Large Codes}}
Code screenshots and Output in C++
\begin{figure}[]
\includegraphics[scale=0.9]{Code 1}
\end{figure}
\begin{figure}[]
\includegraphics[scale=1.2]{Code 2}
\end{figure}
\begin{figure}[]
\includegraphics[scale=1.2]{Code 3}
\end{figure}
\begin{figure}[]
\includegraphics[scale=1.2]{Code 4}
\end{figure}
\begin{figure}[]
\includegraphics[scale=1.2]{Code Output}
\end{figure}
\end{center}
\begin{center}
\pagebreak
Code screenshots and Output in JAVA
\begin{figure}[]
\includegraphics[scale=0.4]{Java 1}
\end{figure}
\begin{figure}[]
\includegraphics[scale=0.4]{Java 2}
\end{figure}
\begin{figure}[]
\includegraphics[scale=0.4]{Java 3}
\end{figure}
\begin{figure}[]
\includegraphics[scale=0.4]{Java 4}
\end{figure}
\begin{figure}[]
\includegraphics[scale=0.4]{Java 5}
\end{figure}
\begin{figure}[]
\includegraphics[scale=0.4]{javaoutput}
\end{figure}
\end{center}

\pagebreak
\pagebreak
\begin{center}
\textbf{\Large Profiler Report and Debugging}
\begin{figure}[b]
\includegraphics[scale=1.2]{process}
\end{figure}
\pagebreak
\pagebreak
\begin{figure}[]
\includegraphics[scale=1]{Output 1}
\end{figure}
\begin{figure}[]
\includegraphics[scale=1.2]{Output 2}
\end{figure}
\begin{figure}[]
\includegraphics[scale=1.2]{Output 3}
\end{figure}
\begin{figure}[!htb]
\includegraphics[scale=1.2]{Output 4}
\end{figure}
\end{center}
\pagebreak

\begin{figure}[]
\includegraphics[scale=1.2]{gdb1}
\end{figure}
\pagebreak
\begin{figure}[!htb]
\includegraphics[scale=1]{gdb2}
\end{figure}
\pagebreak



\end{document}
